\documentclass{beamer}
%
% Choose how your presentation looks.
%
% For more themes, color themes and font themes, see:
% http://deic.uab.es/~iblanes/beamer_gallery/index_by_theme.html
%
\mode<presentation>
{
  \usetheme{Warsaw}      % or try Darmstadt, Madrid, Warsaw, ...
  \usecolortheme{default} % or try albatross, beaver, crane, ...
  \usefonttheme{default}  % or try serif, structurebold, ...
  \setbeamertemplate{navigation symbols}{}
  \setbeamertemplate{caption}[numbered]
} 

\usepackage[polish]{babel}
\usepackage[utf8]{inputenc}
\usepackage[T1]{fontenc}
\usepackage{lineno}
\linenumbers

\newcommand\tab[1][1cm]{\hspace*{#1}}

\title[Nanozombie]{Nanozombie - algorytm}
\author{Marcin Pastwa, Piotr Tomaszewski}
\date{}
\begin{document}

\begin{frame}
  \titlepage
\end{frame}

% Uncomment these lines for an automatically generated outline.
%\begin{frame}{Outline}
%  \tableofcontents
%\end{frame}

\section{Struktury}

\begin{frame}{QUEUE\_PONY}
\internallinenumbers
    \resetlinenumber[1]

    Lista, na której proces przechowuje identyfikatory procesów, które prosiły go o dostęp do sekcji krytycznej (stroju kucyka), gdy ten proces się w niej znajdował (algorytm Ricarta-Agrawali).

    \vspace{1cm}
    Procesy z tej listy są informowane, gdy strój jest zwracany  (otrzymują wiadomość ACK\_PONY).
\end{frame}

\begin{frame}{QUEUE\_SUBMAR\{id\_łodzi\}}
    \internallinenumbers
    \resetlinenumber[1]
    Każdy proces posiada kolejkę, w której znajdują się procesy ubiegające się o dostęp do danej łodzi podwodnej.
    
    \vspace{1cm}
    Na jej podstawie można wyznaczyć, które procesy mogą zająć miejsce na danej łodzi podwodnej (znajdują się w strefie krytycznej).

    \vspace{1cm}
    Kolejka jest bardzo podobna do tej z alg. Lamporta, z tą różnicą, że w sekcji krytycznej na raz może znajdować się >1 proces. Dokładny opis znajduje się w części algorytmicznej.
\end{frame}

\begin{frame}{LIST\_SUBMAR}
    \internallinenumbers
    \resetlinenumber[1]
    Proces przechowuje informację o każdej z łodzi o tym czy, jego zdaniem, znajduje się ona obecnie w porcie, czy też nie.
    TODO: Również czy nie pełna
    
    \vspace{1cm}
    Proces preferuje wybór łodzi stojących w porcie. Jest to próba minimalizacji czasu, który dana łódź spędzi nieużywana.

    \vspace{1cm}
    Zawartość tej listy może być nieaktualna i nie spowoduje to błędu. W najgorszym wypadku, proces zacznie ubiegać się o łódź, która jest w podróży i będzie musiał się wycofać (ALBO CZEKAĆ TODO) i wybrać inną.
\end{frame}

\begin{frame}{DICT\_TOURISTS\_SIZES}
    \internallinenumbers
    \resetlinenumber[1]
    Tablica poglądowa lub w ogólności słownik, w którym kluczem jest identyfikator procesu, natomiast wartością jest rozmiar turysty (ile miejsc na łodzi zajmuje). Wartości te są stałe, więc na potrzeby algorytmu przyjmujemy, że są już każdemu procesowi znane.
    
    \vspace{1cm}
    Jeśli jednak założyć, że wartości te nie są znane z góry, procesy musiałyby wymienić się nimi między sobą przed rozpoczęciem pętli głównej.
\end{frame}

\begin{frame}{DICT\_SUBMAR\_CAPACITY}
    \internallinenumbers
    \resetlinenumber[1]
    Tablica poglądowa lub w ogólności słownik, w którym kluczem jest identyfikator łodzi podwodnej, natomiast wartością jest jej maksymalna pojemność. Wartości te są stałe, więc na potrzeby algorytmu przyjmujemy, że są już każdemu procesowi znane.
\end{frame}

\section{Wiadomości}
\begin{frame}{Znacznik Lamporta (Timestamp)}
    \internallinenumbers
    \resetlinenumber[1]
    W celu określenia relacji uprzedniości zdarzeń w algorytmie zastosowany jest zegar logiczny Lamporta.

    \vspace{0.7cm}
    Do wysyłanych przez proces wiadomości dołączane są znaczniki czasowe.
    
    \vspace{0.7cm}
    Uznaliśmy, że uwzględnienie aktualizacji zegarów i przesyłu znaczników jedynie zmniejszyłoby czytelność algorytmu, dlatego zostało w dalszym opisie ograniczone do minimum.

    \vspace{0.7cm}
    Rozważaliśmy również dołączanie do odpowiedzi na zapytania znacznika tego zapytania. Pozwoliłoby to na weryfikację czy zgoda nie dotyczy jakiegoś przedawnionego zapytania.
\end{frame}

\begin{frame}{REQ\_PONY}
    \internallinenumbers
    \resetlinenumber[1]
    Proces wysyła tę wiadomość, gdy chce uzyskać dostęp do sekcji krytycznej – dostęp do stroju kucyka.
\end{frame}

\begin{frame}{ACK\_PONY}
    \internallinenumbers
    \resetlinenumber[1]
    Stanowi potwierdzenie otrzymania prośby o dostęp do stroju kucyka.

    \vspace{1cm}
    Wysyłana w odpowiedzi na zapytanie REQ\_PONY, gdy odpowiadający proces zgadza się aby pytający uzyskał dostęp do zasobu.
\end{frame}

\section{Stany procesu}
\begin{frame}{RESTING}
    \internallinenumbers
    \resetlinenumber[1]
    Jest to stan początkowy.

    \vspace{1cm}
    Symuluje odpoczynek turysty między zakończeniem jednej wycieczki, a rozpoczęciem kolejnej.

    \vspace{1cm}
    Do kolejnego stanu – WAIT\_PONY – proces przechodzi w pewnym, nieokreślonym momencie. Przyjmujemy, że ten czas jest pewną losową wartością >= 0.
\end{frame}

\begin{frame}{RESTING - odpowiedzi}
    \internallinenumbers
    \resetlinenumber[1]
    Po otrzymaniu REQ\_PONY odpowiada ACK\_PONY.

    \vspace{1cm}
    Po otrzymaniu REQ\_SUBMAR proces dodaje nadawcę do kolejki QUEUE\_SUBMAR\{id\_łodzi\} i odpowiada ACK\_SUBMAR.

    \vspace{1cm}
    ACK\_PONY – ignoruje.

    \vspace{1cm}
    ACK\_SUBMAR – ignoruje.
\end{frame}

\begin{frame}{WAIT\_PONY}
    \internallinenumbers
    \resetlinenumber[1]
    Proces w tym stanie ubiega się o  możliwość zabrania stroju kucyka, czyli na dostęp do sekcji krytycznej.

    \vspace{1cm}
    Do kolejnego stanu – WAIT\_SUBMAR – proces przechodzi po zabraniu stroju kucyka.    
\end{frame}

\begin{frame}{WAIT\_PONY - odpowiedzi}
    \internallinenumbers
    \resetlinenumber[1]
    Na REQ\_PONY odpowiada: \\
        \tab[0.5cm] Jeśli otrzymane zapytanie ma niższy priorytet od wysłanego \\
        \tab[0.5cm] przez ten proces nic nie odpowiada, tylko wstawia id nadawcy 
        \tab[0.5cm] do swojej listy QUEUE\_PONY.
        
        \vspace{0.3cm}
        \tab[0.5cm] W przeciwnym razie, uznaje priorytet rywala i wysyła ACK\_PONY.

    \vspace{0.3cm}
    Proces w tym stanie ubiega się o możliwość zabrania stroju kucyka, czyli na dostęp do sekcji krytycznej.
\end{frame}

\begin{frame}{WAIT\_SUBMAR}
    \internallinenumbers
    \resetlinenumber[1]
    W tym stanie proces ubiega się o dostęp do kolejnej sekcji krytycznej – o zajęcie n miejsc na jednej z łodzi podwodnych.

    \vspace{1cm}
    Do kolejnego stanu – BOARDED – przechodzi, kiedy zajmie zasoby – miejsca na pokładzie.
\end{frame}

\begin{frame}{WAIT\_SUBMAR - odpowiedzi}
    \internallinenumbers
    \resetlinenumber[1]
    REQ\_PONY – nic nie odpowiada, tylko dodaje id nadawcy do swojej listy QUEUE\_PONY.

    \vspace{1cm}
    REQ\_SUBMAR\{id\_łodzi\} – proces dodaje nadawcę do kolejki QUEUE\_SUBMAR\{id\_łodzi\} i odpowiada ACK\_SUBMAR\{id\_łodzi\}. 
\end{frame}

\section{Algorytm}
\begin{frame}{Algorytm}
    \internallinenumbers
    (1.) Proces znajduje się w stanie RESTING.

    \vspace{0.5cm}
    (2.) Po upływie losowo wybranego czasu przechodzi do stanu WAIT\_PONY i zaczyna ubiegać się o dostęp do sekcji krytycznej algorytmem bazującym na alg.Ricarta-Agrawali.

    \vspace{0.5cm}
    (3.) Proces wysyła do pozostałych wiadomość REQ\_PONY i czeka na odpowiedź.
\end{frame}

\begin{frame}{Algorytm}
    \internallinenumbers
    (4.) Każdy proces, który otrzyma REQ\_PONY: \\
        \tab[0.4cm] (a) Jeśli znajduje się w stanie RESTING odpowiada
        \tab[1cm] ACK\_PONY.

        \tab[0.4cm] (b) Jeśli znajduje się w WAIT\_PONY i odebrana\\
        \tab[1cm] wiadomość ma niższy priorytet, niż jego własna, nic nie\\
        \tab[1cm] odpowiada, tylko dodaje nadawcę do\\
        \tab[1cm] QUEUE\_PONY.
        
        \tab[0.4cm] (c) Jeśli znajduje się w WAIT\_PONY i odebrana\\
        \tab[1cm] wiadomość ma wyższy priorytet, uznaje pierwszeństwo  \\
        \tab[1cm] nadawcy i odpowiada ACK\_PONY.

        \tab[0.4cm] (d) Jeśli znajduje się w którymś z pozostałych stanów \\
        \tab[1cm] ma przyznany strój kucyka (Jest w sekcji krytycznej). \\
        \tab[1cm] Nic nie odpowiada, tylko dodaje nadawcę do listy \\
        \tab[1cm] QUEUE\_PONY.
\end{frame}

\begin{frame}{Algorytm}
    \internallinenumbers
    (5.) Tutaj następuje modyfikacja alg. Ricarta - Agrawali. Proces ubiegający się o strój kucyka nie musi czekać na otrzymanie wszystkich potwierdzeń, bo strojów w systemie jest >= 1. Dlatego proces może pobrać strój kucyka, gdy otrzyma \textit{(liczba procesów - liczba strojów)} odpowiedzi ACK\_PONY. Przyjmujemy, że od razuma jedno potwierdzenie - swoje własne.

    \vspace{0.5cm}
    (6.) Po zebraniu wymaganej liczby potwierdzeń proces przechodzi do stanu WAIT\_SUBMAR.

    \vspace{0.5cm}
    (7.) Proces wybiera łódź. Przyjęliśmy, że będzie to pierwsza, która jego zdaniem jest teraz dostępna (LIST\_SUBMAR). Jeśli takiej łodzi nie ma, czeka na otrzymanie RETURN\_SUBMAR\{id\_łodzi, liczba\_pasażerów\} i wybiera tę łódź.
\end{frame}

\begin{frame}{Algorytm}
    \internallinenumbers
    (8.) Proces wysyła do wszystkich pozostałych zapytanie REQ\_SUBMAR\{id\_łodzi\} i dodaje siebie do kolejki QUEUE\_SUBMAR\{id\_łodzi\}.

    \vspace{0.5cm}
    TODO: Spełnić warunek postępu!!!
    (9.) Proces, który otrzymał zapytanie REQ\_SUBMAR\{id\_łodzi\} dodaje nadawcę do kolejki QUEUE\_SUBMAR\{id\_łodzi\} i wysyła odpowiedź ACK\_SUBMAR\{id\_łodzi\}.

\end{frame}

\begin{frame}{Algorytm}
    \internallinenumbers
    W dalszej części algorytmu potrzebny będzie proces, który wyda sygnał do odpłynięcia i potem powrotu. Turyści znajdujący się na łodzi mogliby ubiegać się o dostęp do kolejnej sekcji krytycznej. Jednakże, możemy połączyć tę sekcję z sekcją wsiadania do łodzi i ponownie skorzystając z kolejki QUEUE\_SUBMAR\{id\_łodzi\}, ograniczając tym samym konieczną liczbę przesłanych wiadomości. Zatem sygnał do odpłynięcia i powrotu wyda proces mający pierwszą pozycję w kolejce.
\end{frame}

\begin{frame}{Algorytm}
    \internallinenumbers
    (10.) Po otrzymaniu wszystkich ACK\_SUBMAR proces sprawdza, czy zmieści się na łodzi. Tutaj następuje rozszerzenie alg. Lamporta. Zająć miejsce na łodzi, czyli wejść do sekcji krytycznej może proces, który w kolejce powiązanej z łodzią znajduje się na pozycji $i$, jeśli suma rozmiarów turystów na pozycjach $<= i$ nie przekracza maksymalnej pojemności łodzi. Jeśli się zmieści to zajmuje miejsce, wysyła do pierwszego procesu z kolejki wiadomość TRAVEL\_READY i przechodzi do stanu BOARDED. Jeśli nie, to wysyła do procesów wiadomość FULL\_SUBMAR\{id\_łodzi\}, po czym usuwa się z kolejki, wybiera kolejną łódź i wraca do kroku (8.).
\end{frame}

\begin{frame}{Algorytm}
    \internallinenumbers
    (11.) Procesy, które otrzymały FULL\_SUBMAR\{id\_łodzi\} usuwają nadawcę z kolejki QUEUE\_SUBMAR\{id\_łodzi\} i oznaczają na LIST\_SUBMAR, że dana łódź jest już niedostępna. \\

    (11.a) Proces na pierwszej pozycji w kolejce rozpoczyna przygotowanie do rozpoczęcia podróży. Jeśli sam jeszcze nie zajął zasobów (jest w stanie WAIT\_SUBMAR) odkłada to działanie, aż nie przejdzie do BOARDED. Jeśli jest już w BOARDED sprawdza czy otrzymał już gotowość (TRAVEL\_READY) od pozostałych procesów w sekcji krytycznej. Kiedy już otrzyma wszystkie potwierdzenia wysyła do wszystkich procesów w łodzi wiadomość DEPARTED\_SUBMAR. Czeka, aż wszyscy odpowiedzą ACK\_TRAVEL, po czym wydaje wygnał do odpłynięcia i przechodzi w stan TRAVEL.
\end{frame}

\begin{frame}{Algorytm}
    \internallinenumbers
    (12.) Proces, który otrzyma ACK\_TRAVEL przechodzi w stan TRAVEL i czeka na zakończenie zwiedzania. \\

    (13.) Po pewnym losowym czasie proces informuje pozostałe o zakończeniu podróży. W pierwszej kolejności, wysyła RETURN\_SUBMAR\{id\_łodzi, liczba\_pasażerów\}, do turystów, którzy z nim płynęli (może to stwierdzić patrząc na kolejkę). Chcemy, aby mogli opuścić łódź, nim nowi turyści na nią wsiądą. Po czym zwalnia łódź.
\end{frame}

\begin{frame}{Algorytm}
    \internallinenumbers
    (14.) Procesy, które otrzymały RETURN\_SUBMAR\{id\_łodzi, liczba\_pasażerów\} zwalniają łódź, usuwają z kolejki pierwsze liczba\_pasażerów pozycji, odnotowują przybycie w LIST\_SUBMAR oraz odpowiadają ACK\_TRAVEL. Na końcu przechodzą do stanu TRAVEL\_END.

    (15.) Po otrzymaniu wszystkich potwierdzeń ``kapitan'' wysyła RETURN\_SUBMAR\{id\_łodzi, liczba\_pasażerów\} do pozostałych procesów, informując je, że łódź jest już dostępna. Po czym usuwa pierwsze liczba\_pasażerów pozycji z kolejki. W ten sposób redukujemy liczbę potrzebnych wiadomości. Normalnie, każdy proces zwalniający sekcję krytyczną musiałby poinformować o tym pozostałe. Ponieważ wszyscy turyści w łodzi opuszczają ją w tym samym czasie, to możemy połączyć wszystkie te wiadomości w jedną.
\end{frame}

\begin{frame}{Algorytm}
    \internallinenumbers
    (16.) Proces, który otrzymał RETURN\_SUBMAR\{id\_łodzi, liczba\_pasażerów\} usuwa pierwsze liczba\_pasażerów z kolejki i odnotowuje fakt przybycia łodzi w LIST\_SUBMAR.\\
    (17.) Proces w stanie TRAVEL\_END zwalniają strój kucyka wysyłając ACK\_PONY do wszystkich procesów z QUEUE\_PONY oraz czyści tę listę. Następnie przechodzi do RESTING, czym wraca do kroku (1.).
\end{frame}
\end{document}