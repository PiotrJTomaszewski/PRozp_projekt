\documentclass{beamer}
%
% Choose how your presentation looks.
%
% For more themes, color themes and font themes, see:
% http://deic.uab.es/~iblanes/beamer_gallery/index_by_theme.html
%
\mode<presentation>
{
  \usetheme{Warsaw}      % or try Darmstadt, Madrid, Warsaw, ...
  \usecolortheme{default} % or try albatross, beaver, crane, ...
  \usefonttheme{default}  % or try serif, structurebold, ...
  \setbeamertemplate{navigation symbols}{}
  \setbeamertemplate{caption}[numbered]
} 

\usepackage[polish]{babel}
\usepackage[utf8]{inputenc}
\usepackage[T1]{fontenc}
\usepackage{lineno}
\linenumbers

\newcommand\tab[1][1cm]{\hspace*{#1}}

\title[Nanozombie]{Nanozombie - algorytm}
\author{Marcin Pastwa, Piotr Tomaszewski}
\date{}
\begin{document}

\begin{frame}
  \titlepage
\end{frame}

\section{Stałe, zmienne i struktury}

\begin{frame}{QUEUE\_PONY}
\internallinenumbers
    \resetlinenumber[1]
    Kolejka procesów oczekujących na ACK\_PONY. \\
    Kolejka jest początkowo pusta. \\
    Do kolejki trafiają procesy, które ubiegają się o dostęp do sekcji krytycznej, gdy ten proces się w niej znajduje. Czyli turyści, którzy proszą go o zgodę na pobranie stroju kucyka, gdy ma on przy sobie taki strój. \\
    Po wyjściu z sekcji krytycznej (po zwrocie stroju) do wszystkich procesów z kolejki wysyłane jest ACK\_PONY, kolejka jest następnie czyszczona.
\end{frame}

\begin{frame}{QUEUE\_SUBMAR\{id\_łodzi\}}
    \internallinenumbers
    \resetlinenumber[1]
    Każdy proces posiada po jednej kolejce dla każdej łodzi podwodnej.
    W kolejce znajdują się procesy ubiegające się o miejsce na niej.

    \vspace{0.2cm}
    Wartość ujęta w nawiasy klamrowe oznacza, że mówimy o kolejce powiązanej z łodzią o danym indentyfikatorze.

    \vspace{0.2cm}
    Dzięki kolejce można wyznaczyć, które procesy mogą zająć miejsce na danej łodzi podwodnej (znaleźć się w strefie krytycznej).

    \vspace{0.2cm}
    Kolejka wypełniana jest zgodnie z algorytmem Lamporta, z tą różnicą, że w sekcji krytycznej na raz może znajdować się >1 proces. Dokładniej, proces, który otrzymał wszystkie potwierdzenia i w kolejce ma pozycję ``$i$'', może zająć miejsce (wejść do strefy krytycznej), jeśli suma rozmiarów procesów na pozycjach $<=$``$i$'' jest $<=$ maksymalnej pojemności danej łodzi.
\end{frame}

\begin{frame}{TRY\_NO i MAX\_TRY\_NO}
    \internallinenumbers
    \resetlinenumber[1]
    Aby proces, który czeka w kolejce do łodzi, ale nie starczyło dla niego miejsca nie musiał oczekiwać, aż ta łódź odbędzie podróż i ponownie przybije do brzegu, wycofuje się i próbuje wsiąść do innej łodzi.
    Takie zachowanie stwarza jednak ryzyko, że proces nigdy nie wyruszy na wyprawę. W celu rozwiązania tego problemu proces zlicza w zmiennej TRY\_NO ile razy musiał się wycofać. Zmienna ta jest na początku równa 0 i jest inkrementowana przy każdym wycofaniu. Gdy procesowi uda się zająć miejsce na łodzi zmienna ta jest zerowana. Jeśli jednak liczba prób przekroczy próg MAX\_TRY\_NO, wtedy proces poddaje się i nie próbuje już zmieniać łodzi, tylko zostaje w kolejce do obecnej. \\
    MAX\_TRY\_NO jest parametrem, jego wartość jest pewną liczbą $>=0$. Dokładna wartość tej stałej powinna zostać dobrana eksperymentalnie.
\end{frame}

\begin{frame}{LIST\_SUBMAR}
    \internallinenumbers
    \resetlinenumber[1]
    Lista, w której proces przechowuje informację o każdej łodzi podwodnej, czy jego zdaniem znajduje się ona teraz w porcie, niezapełniona, czy też nie. \\
    Proces preferuje wybór łodzi o najmniejszym stopniu zapełnienia. Wyznaczanie zapełnienia jest złożone obliczeniowo, dlatego jako pewien rodzaj heurystyki przyjmujemy, że proces rozważa tylko te łodzie, które (według jego obecnej wiedzy) stoją w porcie i są niezapełnione. 
    Jak zostało to już podkreślone, zawartość tej listy może być nieaktualna. Nie spowoduje to jednak błędów. W najgorszym wypadku, proces ustawi się w kolejce do łodzi, która odpłynęła i będzie musiał się z niej wycofać i wybrać inną. Jednak, ten problem występuje niezależnie od tego, czy wykorzystamy tę listę, czy nie. Próba uniknięcia tego zjawiska wiązałaby się z ograniczeniem współbieżności.
\end{frame}

\begin{frame}{DICT\_TOURIST\_SIZES}
    \internallinenumbers
    \resetlinenumber[1]
    Tablica poglądowa lub w ogólności słownik, w którym kluczem jest identyfikator procesu, natomiast wartością jest rozmiar turysty (ile miejsc na łodzi zajmuje). 

    \vspace{0.4cm}
    Wartości te są stałe, więc na potrzeby algorytmu przyjmujemy, że są już każdemu procesowi znane.

    \vspace{0.4cm}
    Jeśli jednak założyć, że wartości te nie są znane z góry, procesy musiałyby przesłać swój rozmiar pozostałym przed rozpoczęciem pętli głównej.
\end{frame}

\begin{frame}{DICT\_SUBMAR\_CAPACITY}
    \internallinenumbers
    \resetlinenumber[1]
    Tablica poglądowa lub w ogólności słownik, w którym kluczem jest identyfikator łodzi podwodnej, natomiast wartością jest jej maksymalna pojemność.

    \vspace{0.4cm}
    Wartości te są stałe, więc na potrzeby algorytmu przyjmujemy, że są już każdemu procesowi znane.
\end{frame}


\begin{frame}{Pozostałę stałe i zmienne}
    \begin{description}
        \item [PONY\_NO] Łączna liczba dostępnych strojów kucyka (stała).
        \item [TOURIST\_NO] Łączna liczba turystów (stała).
        \item [SUBMAR\_NO] Łączna liczba łodzi podwodnych (stała).
    \end{description}

\end{frame}

\section{Wiadomości}
\begin{frame}{Znacznik Lamporta (Timestamp)}
    \internallinenumbers
    \resetlinenumber[1]
    Do każdej wiadomości dołączany jest znacznik czasowy (timestamp) modyfikowany zgodnie z zasadami zegara logicznego Lamporta.
\end{frame}

\begin{frame}{Wiadomości}
    \internallinenumbers
    \resetlinenumber[1]
    \begin{description}
        \item [REQ\_PONY] żądanie dostępu do sekcji krytycznej (żądanie dostępu do stroju kucyka).
        \item [ACK\_PONY] potwierdzenie dostępu do stroju kucyka.
        \item [REQ\_SUBMAR\{id\_łodzi\}] żądanie dostępu do sekcji krytycznej (do miejsca na łodzi podwodnej). Id\_łodzi jest parametrem wiadomości.
        \item [ACK\_SUBMAR\{id\_łodzi\}] potwierdzenie wpisania do kolejki do wskazanej łodzi (id\_łodzi).
        \item [FULL\_SUBMAR\_RETREAT\{id\_łodzi\}] wysyłane przez proces, któremu nie udało się wsiąść do łodzi, więc się z niej wycofuje.
        \item [FULL\_SUBMAR\_STAY\{id\_łodzi\}] wysyłane przez proces, któremu nie udało się wsiąść do łodzi, ale poddaje się i zostaje w kolejce.
    \end{description}
\end{frame}

\begin{frame}{Wiadomości c.d.}
    \internallinenumbers
    \begin{description}
        \item [RETURN\_SUBMAR\{id\_łodzi, liczba\_pasażerów\}] informacja, że łódź o podanym id i wioząca podaną liczbę pasażerów wróciła do portu. Dla procesów, które płynęły tą łodzią oznacza to rozkaz zwolnienia zasobów (opuszczenia jej), dla ogółu procesów, że można usunąć z kolejki podaną drugim parametrem liczbę procesów.
        \item [TRAVEL\_READY] informacja, że proces zajął zasoby (wsiadł do łodzi).
    \end{description}

\end{frame}

\begin{frame}{ACK\_PONY}
    \internallinenumbers
    \resetlinenumber[1]
    Stanowi potwierdzenie otrzymania prośby o dostęp do stroju kucyka.

    \vspace{1cm}
    Wysyłana w odpowiedzi na zapytanie REQ\_PONY, gdy odpowiadający proces zgadza się aby pytający uzyskał dostęp do zasobu.
\end{frame}

\section{Stany procesu}
\begin{frame}{RESTING}
    \internallinenumbers
    \resetlinenumber[1]
    Jest to stan początkowy.

    \vspace{1cm}
    Symuluje odpoczynek turysty między zakończeniem jednej wycieczki, a rozpoczęciem kolejnej.

    \vspace{1cm}
    Do kolejnego stanu – WAIT\_PONY – proces przechodzi w pewnym, nieokreślonym momencie. Przyjmujemy, że ten czas jest pewną losową wartością >= 0.
\end{frame}

\begin{frame}{RESTING - odpowiedzi}
    \internallinenumbers
    \resetlinenumber[1]
    Po otrzymaniu REQ\_PONY odpowiada ACK\_PONY.

    \vspace{1cm}
    Po otrzymaniu REQ\_SUBMAR proces dodaje nadawcę do kolejki QUEUE\_SUBMAR\{id\_łodzi\} i odpowiada ACK\_SUBMAR.

    \vspace{1cm}
    ACK\_PONY – ignoruje.

    \vspace{1cm}
    ACK\_SUBMAR – ignoruje.
\end{frame}

\begin{frame}{WAIT\_PONY}
    \internallinenumbers
    \resetlinenumber[1]
    Proces w tym stanie ubiega się o  możliwość zabrania stroju kucyka, czyli na dostęp do sekcji krytycznej.

    \vspace{1cm}
    Do kolejnego stanu – WAIT\_SUBMAR – proces przechodzi po zabraniu stroju kucyka.    
\end{frame}

\begin{frame}{WAIT\_PONY - odpowiedzi}
    \internallinenumbers
    \resetlinenumber[1]
    Na REQ\_PONY odpowiada: \\
        \tab[0.5cm] Jeśli otrzymane zapytanie ma niższy priorytet od wysłanego \\
        \tab[0.5cm] przez ten proces nic nie odpowiada, tylko wstawia id nadawcy 
        \tab[0.5cm] do swojej listy QUEUE\_PONY.
        
        \vspace{0.3cm}
        \tab[0.5cm] W przeciwnym razie, uznaje priorytet rywala i wysyła ACK\_PONY.

    \vspace{0.3cm}
    Proces w tym stanie ubiega się o możliwość zabrania stroju kucyka, czyli na dostęp do sekcji krytycznej.
\end{frame}

\begin{frame}{WAIT\_SUBMAR}
    \internallinenumbers
    \resetlinenumber[1]
    W tym stanie proces ubiega się o dostęp do kolejnej sekcji krytycznej – o zajęcie n miejsc na jednej z łodzi podwodnych.

    \vspace{1cm}
    Do kolejnego stanu – BOARDED – przechodzi, kiedy zajmie zasoby – miejsca na pokładzie.
\end{frame}

\begin{frame}{WAIT\_SUBMAR - odpowiedzi}
    \internallinenumbers
    \resetlinenumber[1]
    REQ\_PONY – nic nie odpowiada, tylko dodaje id nadawcy do swojej listy QUEUE\_PONY.

    \vspace{1cm}
    REQ\_SUBMAR\{id\_łodzi\} – proces dodaje nadawcę do kolejki QUEUE\_SUBMAR\{id\_łodzi\} i odpowiada ACK\_SUBMAR\{id\_łodzi\}. 
\end{frame}

\section{Algorytm}
\begin{frame}{Algorytm}
    \internallinenumbers
    (1.) Proces znajduje się w stanie RESTING.

    \vspace{0.5cm}
    (2.) Po upływie losowo wybranego czasu przechodzi do stanu WAIT\_PONY i zaczyna ubiegać się o dostęp do sekcji krytycznej algorytmem bazującym na alg.Ricarta-Agrawali.

    \vspace{0.5cm}
    (3.) Proces wysyła do pozostałych wiadomość REQ\_PONY i czeka na odpowiedź.
\end{frame}

\begin{frame}{Algorytm}
    \internallinenumbers
    (4.) Każdy proces, który otrzyma REQ\_PONY: \\
        \tab[0.4cm] (a) Jeśli znajduje się w stanie RESTING odpowiada
        \tab[1cm] ACK\_PONY.

        \tab[0.4cm] (b) Jeśli znajduje się w WAIT\_PONY i odebrana\\
        \tab[1cm] wiadomość ma niższy priorytet, niż jego własna, nic nie\\
        \tab[1cm] odpowiada, tylko dodaje nadawcę do\\
        \tab[1cm] QUEUE\_PONY.
        
        \tab[0.4cm] (c) Jeśli znajduje się w WAIT\_PONY i odebrana\\
        \tab[1cm] wiadomość ma wyższy priorytet, uznaje pierwszeństwo  \\
        \tab[1cm] nadawcy i odpowiada ACK\_PONY.

        \tab[0.4cm] (d) Jeśli znajduje się w którymś z pozostałych stanów \\
        \tab[1cm] ma przyznany strój kucyka (Jest w sekcji krytycznej). \\
        \tab[1cm] Nic nie odpowiada, tylko dodaje nadawcę do listy \\
        \tab[1cm] QUEUE\_PONY.
\end{frame}

\begin{frame}{Algorytm}
    \internallinenumbers
    (5.) Tutaj następuje modyfikacja alg. Ricarta - Agrawali. Proces ubiegający się o strój kucyka nie musi czekać na otrzymanie wszystkich potwierdzeń, bo strojów w systemie jest >= 1. Dlatego proces może pobrać strój kucyka, gdy otrzyma \textit{(liczba procesów - liczba strojów)} odpowiedzi ACK\_PONY. Przyjmujemy, że od razu ma jedno potwierdzenie - swoje własne.

    \vspace{0.5cm}
    (6.) Po zebraniu wymaganej liczby potwierdzeń proces przechodzi do stanu WAIT\_SUBMAR.

\end{frame}

\begin{frame}{Algorytm}
    \internallinenumbers
    (7.) Proces wybiera łódź. Przyjęliśmy, że będzie to łódź, która według jego aktualnej wiedzy jest w najmniejszym stopniu zajęta. Wyznaczenie zajętości może być wymagające obliczeniowo, dlatego proces rozważa tylko te łodzie, na które jego zdaniem są jeszcze dostępne (LIST\_SUBMAR). Jeśli takiej łodzi nie mat, to proces czeka na sygnał RETURN\_SUBMAR.
\end{frame}

\begin{frame}{Algorytm}
    \internallinenumbers
    (8.) Proces wysyła do wszystkich pozostałych zapytanie REQ\_SUBMAR\{id\_łodzi\} i dodaje siebie do kolejki QUEUE\_SUBMAR\{id\_łodzi\}.

    \vspace{0.5cm}
    (9.) Proces, który otrzymał zapytanie REQ\_SUBMAR\{id\_łodzi\} dodaje nadawcę do kolejki QUEUE\_SUBMAR\{id\_łodzi\} i wysyła odpowiedź ACK\_SUBMAR\{id\_łodzi\}.

\end{frame}

\begin{frame}{Algorytm}
    \internallinenumbers
    W dalszej części algorytmu potrzebny będzie proces, który wyda sygnał do odpłynięcia i potem powrotu. Turyści znajdujący się na łodzi mogliby ubiegać się o dostęp do kolejnej sekcji krytycznej. Jednakże, możemy połączyć tę sekcję z sekcją wsiadania do łodzi i ponownie skorzystając z kolejki QUEUE\_SUBMAR\{id\_łodzi\}, ograniczając tym samym konieczną liczbę przesłanych wiadomości. Zatem sygnał do odpłynięcia i powrotu wyda proces mający pierwszą pozycję w kolejce.
\end{frame}

\begin{frame}{Algorytm}
    \internallinenumbers
    (10.) Po otrzymaniu wszystkich ACK\_SUBMAR proces sprawdza, czy zmieści się na łodzi. Tutaj następuje rozszerzenie alg. Lamporta. Zająć miejsce na łodzi, czyli wejść do sekcji krytycznej może proces, który w kolejce powiązanej z łodzią znajduje się na pozycji $i$, jeśli suma rozmiarów turystów na pozycjach $<= i$ nie przekracza maksymalnej pojemności łodzi. Jeśli się zmieści to zajmuje miejsce, wysyła do pierwszego procesu z kolejki wiadomość TRAVEL\_READY i przechodzi do stanu BOARDED. Jeśli nie, sprawdza czy przekroczył już maksymalną liczbę prób, jeśli tak to się poddaje i stwierdza, że poczeka sobie w kolejce. Wysyła wtedy do procesów FULL\_SUBMAR\_STAY\{id\_łodzi\}. W przeciwnym razie wysyła do procesów wiadomość FULL\_SUBMAR\_RETREAT\{id\_łodzi\}, po czym usuwa się z kolejki. \\
    Wybiera kolejną łódź i wraca do kroku (8.).
\end{frame}

\begin{frame}{Algorytm}
    \internallinenumbers
    (11.) Procesy, które otrzymały FULL\_SUBMAR\_RETREAT\{id\_łodzi\} usuwają nadawcę z kolejki QUEUE\_SUBMAR\{id\_łodzi\} i oznaczają na LIST\_SUBMAR, że dana łódź jest już niedostępna. Jeśli była to wiadomość FULL\_SUBMAR\_STAY\{id\_łodzi\} jedynie oznaczają łódź jako niedostępną. \\

    (11.a) Proces na pierwszej pozycji w kolejce rozpoczyna przygotowanie do rozpoczęcia podróży. Jeśli sam jeszcze nie zajął zasobów (jest w stanie WAIT\_SUBMAR) odkłada to działanie, aż nie przejdzie do BOARDED. Jeśli jest już w BOARDED sprawdza czy otrzymał już gotowość (TRAVEL\_READY) od pozostałych procesów w sekcji krytycznej. 
    ????TODO:Kiedy już otrzyma wszystkie potwierdzenia wysyła do wszystkich procesów w łodzi wiadomość DEPART\_SUBMAR. Czeka, aż wszyscy odpowiedzą ACK\_TRAVEL, po czym wydaje wygnał do odpłynięcia i przechodzi w stan TRAVEL.
\end{frame}

\begin{frame}{Algorytm}
    \internallinenumbers
    (12.) Proces, który otrzyma ACK\_TRAVEL przechodzi w stan TRAVEL i czeka na zakończenie zwiedzania. \\

    (13.) Po pewnym losowym czasie proces informuje pozostałe o zakończeniu podróży. W pierwszej kolejności, wysyła RETURN\_SUBMAR\{id\_łodzi, liczba\_pasażerów\}, do turystów, którzy z nim płynęli (może to stwierdzić patrząc na kolejkę). Chcemy, aby mogli opuścić łódź, nim nowi turyści na nią wsiądą. Po czym zwalnia łódź.
\end{frame}

\begin{frame}{Algorytm}
    \internallinenumbers
    (14.) Procesy, które otrzymały RETURN\_SUBMAR\{id\_łodzi, liczba\_pasażerów\} zwalniają łódź, usuwają z kolejki pierwsze liczba\_pasażerów pozycji, odnotowują przybycie w LIST\_SUBMAR oraz odpowiadają ACK\_TRAVEL. Na końcu przechodzą do stanu ON\_SHORE.

    (15.) Po otrzymaniu wszystkich potwierdzeń ``kapitan'' wysyła RETURN\_SUBMAR\{id\_łodzi, liczba\_pasażerów\} do pozostałych procesów, informując je, że łódź jest już dostępna. Po czym usuwa pierwsze liczba\_pasażerów pozycji z kolejki. W ten sposób redukujemy liczbę potrzebnych wiadomości. Normalnie, każdy proces zwalniający sekcję krytyczną musiałby poinformować o tym pozostałe. Ponieważ wszyscy turyści w łodzi opuszczają ją w tym samym czasie, to możemy połączyć wszystkie te wiadomości w jedną.
\end{frame}

\begin{frame}{Algorytm}
    \internallinenumbers
    (16.) Proces, który otrzymał RETURN\_SUBMAR\{id\_łodzi, liczba\_pasażerów\} usuwa pierwsze liczba\_pasażerów z kolejki i odnotowuje fakt przybycia łodzi w LIST\_SUBMAR.\\
    (17.) Proces w stanie ON\_SHORE zwalniaja strój kucyka wysyłając ACK\_PONY do wszystkich procesów z QUEUE\_PONY oraz czyści tę listę. Następnie przechodzi do RESTING, czym wraca do kroku (1.).
\end{frame}
\end{document}